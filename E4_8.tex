\documentclass[a4paper]{article}
\usepackage{geometry}
\geometry{left=2.5cm,right=2.5cm}
\usepackage[UTF8]{ctex}
\usepackage{algorithm}
\usepackage{algpseudocode}
\usepackage{amsmath}
\usepackage{amssymb}

\def\bm{\boldsymbol}
\floatname{algorithm}{习题}
\algnewcommand\Process{\item[\textbf{过程:}]}
\algnewcommand\Or{\ \textbf{OR}\ }
\algnewcommand\Break{\textbf{break}}
\algnewcommand\Continue{\textbf{continue}}
\algrenewcommand{\algorithmicrequire}{\textbf{输入:}}
\algrenewcommand{\algorithmicensure}{\textbf{输出:}}

\begin{document}

\renewcommand{\thealgorithm}{4.8}
\begin{algorithm}[H]
    \caption{以参数$\mathtt{MaxNode}$控制最大深度的BFS决策树算法}
    \begin{algorithmic}[1]
        \Require{训练集~$D=\{(\bm x_1,y_1),(\bm x_2,y_2),\cdots,(\bm x_m,y_m)\}$;

属性集~$A=\{a_1,a_2,\cdots,a_d\}$;

最大结点数~$\mathtt{MaxNode}$。}
        \Process{函数~$\mathtt{TreeGenerate\_BFS\_MaxNode}(D, A, \mathtt{MaxNode})$}
        
        \State{创建空队列~$\mathtt{queue}$;}
        \State{生成结点~$\mathtt{root}$;}
        \State{将~$(\mathtt{root}, D, A)$~入~$\mathtt{queue}$~队尾;}
        \State{$\mathtt{count}\gets1$;}\Comment\textit{初始化结点个数}

        \While{$\mathtt{queue}$~不为空}
            \State{从~$\mathtt{queue}$~队头出队~$(\mathtt{node}, D_\mathrm h, A_\mathrm h)$;}

            \If{$D_\mathrm h$~中样本全属于同一类别~$C$}
                \State{将~$\mathtt{node}$~标记为~$C$~类叶结点; }\Continue
            \EndIf

            \If{$A_\mathrm h=\varnothing$\Or$D_\mathrm h$~中样本在~$A_\mathrm h$~上取值相同}
                \State{将~$\mathtt{node}$~标记为叶结点,其类别标记为~$D_\mathrm h$~中样本数最多的类; }\Continue
            \EndIf

            \State{从~$A_\mathrm h$~中选择最优划分属性~$a^*$;}
            \For{$a^*$~的每一个值~$a^*_v$}
                \State{为~$\mathtt{node}$~生成一个分支~$\mathtt{child}_v$; 令~$D_v$~表示~$D_\mathrm h$~中在属性~$a^*$~上取值为~$a^*_v$~的样本子集;}
                \State{$\mathtt{count} \gets \mathtt{count}+1$;}

                \If{$\mathtt{count} \ge \mathtt{MaxNode}$}
                    \State{将~$\mathtt{child}_v$~标记为叶结点,其类别标记为~$D_\mathrm h$~中样本数最多的类; }\Break\ \algorithmicwhile
                \EndIf

                \If{$D_v$~为空}
                    \State{将~$\mathtt{child}_v$~标记为叶结点,其类别标记为~$D_\mathrm h$~中样本数最多的类; }
                \Else
                    \State{将~$(\mathtt{child}_v, D_v, A \setminus \{a^*\})$~入~$\mathtt{queue}$~队尾;}
                \EndIf
            \EndFor
        \EndWhile

        \Ensure{以~$\mathtt{root}$~为根结点的一颗决策树}
    \end{algorithmic}
\end{algorithm}

二者均为~BFS~算法。仅从内存角度考虑,$\mathtt{MaxDepth}$~的策略需要对队列中每个结点都多存储一个深度值,且决策树广度较大时可能无法较好地限制结点数量的增长,而~$\mathtt{MaxNode}$~策略从结点数量上限制,能同时限制深度和广度的增长,故~$\mathtt{MaxNode}$~策略更佳。

\end{document}
